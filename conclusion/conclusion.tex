\section{Conclusion}

Overall, the project was a success. The team as a whole had the opportunity to use tools that allowed us to streamline our work quite effectively. As some members were more familiar than others on a particular tool or technology, communication was essential. We relied heavily on \texttt{bitbucket.org} to store code, track and assign issues to each other. A post-meeting email was systematically sent with goals for each member of the team, even when all team members were physically present at the meeting.

The final sprints were conducted when all team members were physically separated in locations that spanned 6 different timezones (from the East Coast USA to Poland). This proves that while distance certainly can hinder efficiency, remote development teams can still function well. Correct and up-to-date commit messages, distributed version control, and frequent communication were key to overcoming glitches and ensuring our success.

In this project we have learned about how \texttt{DIFF} and \texttt{PATCH} tools work, and why they are so important. We have also gained experience in working with clone detection techniques such as suffix trees, which proved a very effective addition to the \texttt{diffr} tool. The tools we produced performed well against the oft-used \texttt{GNU DIFF}; the result was a slightly slower, but well scaling tool that outputs significantly smaller patch files. As an additional challenge, the group plans to port this tool to \texttt{C/C++} over the summer, in order to improve the runtime.
