\chapter{Requirements}

Here I will list the functional and non-functional requirements for this tool.
Each requirement listing consists of 4 columns: ID, Requirement, Priority and Risk.
These columns are explained separately below.

\paragraph{ID}
The identifier of the requirement.
Functional requirement IDs consist of an \emph{F} followed by a number, whilst non-functional requirement IDs consist of \emph{N} followed an number.

\paragraph{Requirement}
The description of the requirement.
This column aims to concisely describe what is required, and consists of a statement about what the system shall do.

\paragraph{Priority}
The priority of the requirement.
This column gives a qualitative judgement of the determined priority, and is ranked M for \emph{must do}, S for \emph{should do} and C for \emph{could do}.
The project's conclusion should see all requirements ranked M completed and most requirements ranked S completed.

\paragraph{Risk}
The risk of the requirement.
This column gives a qualitative judgement of the determined risk, in terms of time, effort, effect on other requirements and likelihood of failure.
Possible ranks are High, Medium and Low with obvious definitions.

\section{Functional Requirements}

Here I will list the functional requirements for the tool.
These requirements describe something the tool should do.

\begin{center}
\begin{longtable}{c p{2.8in} c c }

	\toprule
	\textbf{ID} & \multicolumn{1}{c}{\textbf{Requirement}} & \textbf{Priority} & \textbf{Risk} \\
	\midrule
	
	F01 & The System shall take as input two plain text files. & M & Low \\
	F02 & The System shall output plain text. & M & Low \\
	F03 & The System shall output to standard out. & M & Low \\
	F04 & The System shall compute the differences between the two input files. & M & Medium \\
	F05 & The System shall identify which sections are copied from the first file to the second. & M & High \\

\bottomrule        
\end{longtable}
\end{center}

\section{Non-Functional Requirements}

Here I will list the non-functional requirements for the tool.
These requirements describe how the system should be.

\begin{center}
\begin{longtable}{c p{2.8in} c c }

	\toprule
	\textbf{ID} & \multicolumn{1}{c}{\textbf{Requirement}} & \textbf{Priority} & \textbf{Risk} \\
	\midrule
	
	N01 & The System shall scale well to large documents. & S & Medium \\
	N02 & The System shall process files up to $10,000$ lines long within seconds. & S & Medium \\
	N03 & The System shall be cross-platform. & C & Medium \\
	N04 & The System shall use a suffix tree in order to detect clones between two documents. & C & High \\
	
\bottomrule
\end{longtable}
\end{center}
