\section{Introduction}

%The main task for this group coursework is to build a DIFF tool together with a PATCH tool that is copy and move aware.
%\href{http://www.gnu.org/software/diffutils/}{GNU DIFF} works line by line and determines the differences between two text files, producing a sequence of commands that can be saved to a DIFF file.
%\href{http://www.gnu.org/software/diffutils/}{GNU PATCH} can then read these commands and apply them to the first file, which will recreate the second file input to DIFF.
%Commands either copy an ordered set of lines from the first file, delete an ordered set of lines from the first file, or insert a new set of lines.
%
%An issue with this is that if a line is copied numerous times throughout the second file, a command must insert it every time.
%This is inefficient, and can result in a large DIFF file.
%%Specific line sequences are either inserted or deleted by a command.
%%DIFF identifies an insertion and deletion in the same area as a change of line sequence.
%%
%%Our DIFF tool works differently than the known DIFF.
%%It has the following new features such as finding the longest segments that are copied or moved from one file to the other file.
%%It shows lines that are in the new version, but not in the old version.
%%It shows if lines are in the new version that is also present in the old.
%%But instead of showing the lines, it shows where the lines in the old files are.
%
%PATCH tool is the automated process of committing or applying of changes in files. 
%This is necessary as many people can work on different copies of the same file.
%The parts they changed are written in the form of DIFF output.
%Hence, PATCH tool takes a DIFF output and applies its commands on a file.
%The PATCH commands are insert, delete and change lines. 
%
%PATCH tool example: A file is created called DIFFS that includes the differences between two files, File1 and File2 as DIFF output.
%This file can be shared with other people now.
%Then, they can decide whether they want to commit the changes from copy of File1 or not.
%After they commit the change, PATCH applies the changes in DIFFS on File2.
%Hence, both File1 and File2 are identical now.
%If the DIFF output includes changes from several files, PATCH can process and apply them.
%
%Our PATCH tool works similar to the above mentioned known PATCH tool.
%But it uses an old version and an output of our DIFF tool to produce the new version from it.
%Hence, this DIFF/PATCH tool can be used to determine and distribute changes easily and effectively.
%
%In this report, the requirements of this tool are identified.
%UML diagrams are included that shows the architecture of the detector.
%The description about the implementation of this tool, how it works and the testing performed are included.
%Finally, the results produced by this tool are evaluated including the main things learnt from this coursework.

\href{http://www.gnu.org/software/diffutils/}{GNU DIFF} is a tool that allows a programmer to identify the differences between two files, and outputs a list of commands to transform the first into the second.
These commands can be saved and used as input to GNU PATCH, together with the first file, in order to perform the transformation.
This is basic change control, and it revolutionised both single and multi-user programming, allowing file differences to be exchanged as an alternative to complete modified files.

There are three DIFF commands: copy a set of lines from the first file, delete a set of lines from the first file, or insert a set of lines.
The commands in the patch file need to be applied in order, otherwise the transformation will be incorrect. Moreover, due to the  nature of the instructions, every line in the original file needs to be handled explicitly, resulting in more instructions and hence longer patch files.

Clone detection is the process of identifying matching sections of text between two files.
Much research is done in the field to enhance its application in areas such as plagiarism detection and intellectual property theft.
As a result, much faster methods are available today to detect matching sections of text than were available when GNU DIFF was designed.
In particular, the Suffix Tree method allows for an extremely fast detection algorithm.

Our aim is to combine these two concepts, and create a new pair of tools: \texttt{diffr} and \texttt{patchr}.
These tools will be copy and move aware, and support out-of-order clone detection.
As a result,only two commands are necessary: copy a set of ordered lines from the first file, or insert a set of lines.
The tools will be extremely efficient, and the output from \texttt{diffr} will contain the bare minimum that is needed to articulate the set of differences between the files.
