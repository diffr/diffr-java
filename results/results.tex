\section{Results}

The results achieved by our system are evaluated in this section and compared against \texttt{GNU DIFF}, already described in \Cref{Introduction}. 

\subsection{Test files}
The test files that we used for generating these results are the same as those used for integration testing (\Cref{Testing}). The lengths of the files used are as follows: 

\begin{table}[htbp]
\begin{center}
\begin{tabular}{ l l }
\textbf{File} & \textbf{Length (lines)} \\
\cmidrule(r){1-2}
\texttt{version 2.6.27.62} & 106086 \\
\texttt{version 3.2.13}    & 177408 \\
\end{tabular}
\end{center}
\caption{Lengths (in lines) of the test files used for comparing the performance of our tool and \texttt{GNU DIFF}.}
\end{table}

\subsection{Experimental setup} 
When measuring the performance, we perform $3$ test runs, in order to heat up disk caches and not give unfair advantage to either tool, followed by $10$ experimental runs.

\subsection{Results of \texttt{diffr} and \texttt{patchr}}

\subsubsection*{Time taken}
\begin{description}
\item[Command] \texttt{time ./diffr.sh kernel26.txt kernel33.txt -o kernel.patch}
\begin{table}[htbp]
\begin{center}
\begin{tabular}{ l p{3in} }
\textbf{Measurement} & \textbf{Times [s]} \\
\cmidrule(r){1-2}
real & 1.852, 1.875, 1.806, 2.130, 1.927, 1.806, 2.135, 1.949, 2.016, 1.835 \\
\end{tabular}
\end{center}
\caption{Experiment runs of our tool.}
\end{table}

\end{description}

\subsubsection*{Patch length}
\begin{description}
\item[Command] \texttt{cat kernel.patch | wc -l}
\item[Result] 101282
\end{description}
  
\subsection{Results of \texttt{GNU DIFF} }
\subsubsection*{Time taken}
\begin{description}
\item[Command] \texttt{time diff kernel26.txt kernel33.txt > kernel.patch}
\begin{table}[htbp]
\begin{center}
\begin{tabular}{ l p{3in} }
\textbf{Measurement} & \textbf{Times [s]} \\
\cmidrule(r){1-2}
real & 0.673, 0.634, 0.693, 0.695, 0.713, 0.686, 0.677, 0.660, 0.674, 0.695 \\
\end{tabular}
\end{center}
\caption{Experiment runs of \texttt{GNU DIFF}.}
\end{table}

\end{description}

\subsubsection*{Patch length}
\begin{description}
\item[Command] \texttt{cat kernel.patch | wc -l}
\item[Result] 254830
\end{description}

\subsection{Evaluation}
\texttt{GNU DIFF} produced patch files almost twice as large as those produced by our tool.
Our system is clearly, for this set of test files, much more efficient in terms of patch file length. 

Given that both samples are normally distributed (\texttt{Kolmogorov-Smirnov test; \texttt{diffr} $p=0.431211$; \texttt{GNU DIFF} $p=0.62061$}), we can use a parametric t-test to evaluate the performance.
This test concludes that the mean run time of our tool is significantly higher than the mean run time of GNU DIFF at 5\% confidence level ($p=5.60009\times 10^{-11}$).
Therefore, given the means ($\mbox{diffr} = 1.9331$, $\mbox{GNU DIFF}=0.68$), we can conclude that our system is slower for the set of test files used for the evaluation.
However, we believe this is due to the overhead of the Java Virtual Machine startup, which is significant for such short-running programs.
