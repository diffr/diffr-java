\section{Team}

Here I will discuss the structure of the team, and the responsibilities of each member on this project.

\subsection{Development Process}

We used the scrum development cycle to structure our project.
This meant dividing the work into a series of \emph{sprints}, which each lasted approximately half a week.
Usually sprint lengths are in the order of weeks, but there was a short time to complete this project, and so short iterations were essential to a timely conclusion.

Each sprint was started with a meeting: either physical, or after the end of term, electronic.
In this meeting we reviewed the progress of the tasks of the previous sprint, deciding what could be released, and what must roll over to the next sprint.
Following the release of the previous sprint, we decided the present sprint's deadline, and discussed tasks which need doing.
Tasks were then allocated, first by preference, and then arbitrarily.

\subsection{Structure}

The structure of the team was equal and democratic; we followed the scrum development process to decide tasks and deadlines, and would discuss any issues as and when they presented themselves.
Due to the small size and time to complete the project, role allocation would have adversely affected the progress of the project.

\subsection{Responsibilities}

As discussed above, each team member assumed an equal role within the group, and we all had a joint responsibility complete the project on time.
This translated into a responsibility to complete our tasks on time and to the best of our ability, and to take an active role in scrum meetings.
Meetings were crucial to the success of the project: ensuring that the right tasks were set and allocated to the right members, as well as ensuring that the deadlines were both feasible and on track for the project submission deadline.
